\section{Políticas de Operación Aplicables}

\begin{itemize}
	\item PO1 La correspondencia que ingrese, dirigida al Director del Centro Mexicano para la Producción más Limpia (CMP+L), deberá ser registrada y controlada, a través del control de correspondencia de la intranet.
	\item PO2 Los documentos que ingresen al Centro Mexicano para la Producción más Limpia, deberán estar firmados por el remitente y con el sello correspondiente de la entidad que envía; si contiene anexo archivo magnético, se comprobará que contenga la información descrita en el documento; si menciona anexos, se verificará que estén completos.
	\item PO4 El documento original deberá escanearse y subirse al apartado de oficios recibidos dentro de la aplicación de control de correspondencia de la intranet.
\end{itemize}

\section{Reglas de Negocio}

\begin{itemize}
	\item[RN1] Se considerará como correspondencia todo documento que quede registrado en SISA junto con su copia escaneada.
	\item[RN2] Sólo serán registrados los documentos de correspondencia que estén firmados por el remitente y con sello correspondiente de la entidad que envía.
	\item[RN3] Sólo los usuarios registrados en el directorio podrán hacer uso de la aplicación web.
	\item[RN4] No debe existir correspondencia con el mismo número de oficio.
	\item[RN5] La oficialía de partes terminará con el registro de la correspondencia saliente.
	\item[RN6] Cada usuario de la aplicación, a excepción del administrador y la oficialía de partes, deberá ver sólo la correspondencia que le concierne.
	\item[RN7] La correspondencia puede no llevar anexos. 
	\item[RN8] La correspondencia entrante solo puede ser turnada por primera vez al director.
	\item[RN9] La correspondencia entrante será turnada por el director al personal que debe atender el asunto.
	\item[RN10] La oficialía de partes tiene que agregar a la aplicación un documento escaneado del acuse de la correspondencia que salió del CMPL.
	\item[RN11] Una vez escaneado y turnado el documento de correspondencia entrante, éste documento se debe archivar.
	\item[RN12] La oficialía de partes será el único usuario con el privilegio de cancelar el seguimiento de la correspondencia.
	\item[RN13] Si el turnado de un oficio entrante es público, se turna el oficio a todo el personal con un cargo en el área.
	\item[RN14] Si el turnado de un oficio es privado, sólo se le turna al encargado del área.
	\item[RN15] Si se turna un oficio por área, el responsable del área debe emitir una respuesta.
	\item[RN16] La respuesta a un oficio debe ser emitida dentro de la fecha límite de respuesta.
	\item[RN17] Si un oficio se turna a una persona en específico, sólo esta persona debe emitir una respuesta.
	\item[RN18] Sólo la subdirección Técnica puede turnar oficios entrantes a las dos jefaturas a su cargo.
	\item[RN19] Sólo la oficialía de partes puede registrar oficios entrantes.
	\item[RN20] Los oficios entrantes registrados por la oficialía de partes deben ser turnados al director.
	\item[RN21] Sólo el director puede turnar oficios por área.
	\item[RN22] Sólo el director puede turnar oficios en atención directa a una persona.
	\item[RN23] Las dependendencias sólo pueden ser registradas en caso de que no exista en la aplicación.
	\item[RN24] Las entidades externas sólo pueden ser registradas en caso de que no existan en la aplicación.
	\item[RN25] Las áreas de las dependencias externas sólo podrán ser registradas en caso de que no existan en la aplicación.
	\item[RN26] Los cargos de las entidades externas sólo podrán ser registrados en caso de que no existan en la aplicación.
	\item[RN27] Sólo se registrará correspondencia entrante si tiene su respectivo oficio.
	\item[RN28] Por normatividad no se recibe correspondencia sin oficio.
	\item[RN29] Se recibirá correspondencia sin oficio si se le conoce su procedencia, es indispensable para las actividades del CMPL y es aprobada su recepción por las autoridades correspondientes.
	\item[RN30] La correspondencia recibida sin oficio se registra como correspondencia especial.
	\item[RN31] Cuando un oficio entrante está en estado de registrado, sólo podrá ser visualizado por Oficialía de Partes y el Director.
	\item[RN32] Si un oficio entrante no tiene fecha límite de respuesta, la aplicación asigna una fecha límite de tres semanas asignado el oficio con prioridad “Baja”.
	\item[RN33] Sólo el administrador puede crear nuevos usuarios.
	\item[RN34] Al registrar un nuevo usuario todos los campos del formulario deben de estar llenos.
	\item[RN35] Solo se puede eliminar un usuario si no tiene documentos pendientes asociados a él.
	\item[RN36] Cada usuario debe tener un registro único dentro de la aplicación.
	\item[RN37] Sólo el director puede turnar copia de oficios.
	\item[RN38] El turnado de las copias sólo podrá ser enviado a los encargados de las diferentes áreas del CMPL
	\item[RN39] Oficialía de Partes debe validar el formato de un oficio saliente antes de ser firmado.
\end{itemize}
