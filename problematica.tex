En este capítulo abordamos la problemática que presenta el CMPL en la actualidad.

\section{Problemática}

%Se describe el contenido del capítulo
En el presete capítulo identificamos, describimos y analizamos la problemática del trabajo terminal. Para realizarlo seguimos un procedimiento de descomposición de la problemática, identificación de causas y la integración de posibles soluciones. Al final del capítulo presentamos la estrategia que se utilizará para el desarrollo del trabajo terminal. 
\section{Planteamiento del problema}
%En un párrafo se presenta y describe la problemática en general con el contexto.
Como mencionamos al final del capítulo anterior, en la actualidad el CMPL realiza todas sus actividades de forma manual. Llevar los procesos de forma manual tiene un impacto negativo en la calidad con la que el CMPL trabaja, ya que resulta complicado llevar los procesos conforme a los manuales. La ausencia de una herramienta de apoyo que reduzca el impacto ambiental en el cmpl, ha sido la causa principal de que el CMPL haya tenido problemas en la obtención de la certificación de la norma ISO 14001. 

\section{Análisis del Problema}
%Se explica el procedimiento para el análisis de la problemática.
Los problemas fueron detectados con base a las reuniones con el cliente. Una vez detectados los problemas, describimos cada uno con el fin de identificar sus causas y el impacto que éstos traen al CMPL. Finalmente se van a proponer estrategias a implementar con las posibles soluciones a los problemas. 
	\subsection{Descomposición del problema}
	%Se presentan y describen los problemas uno por uno y se identifica la relación entre ellos.
	A continuación se mencionan los problemas actualmente detectados en el CMPL. 
		\subsubsection{Problemas con los oficios y memorándums}
\begin{itemize}
	\item No siempre se le informa al destinatario la llegada de sus oficios o memorádums.
	\item Los oficios con información confidencial es propensa a estar expuesta (violación de la privacidad).
	\item No se sabe el estatus del seguimiento de los oficios.
	\item Es complicado consultar la información de los oficios que están archivados.
	\item El registro de los oficios no se realiza de forma correcta.
	\item Rezago tecnológico.
	\item Se atienden oficios fuera de la fecha límite de respuesta.
	\item No se sabe el desempeño de trabajo del centro respecto a la atención de asuntos con dependecias externas.
	\item Es complicado saber la ubicación física de algunos oficios o memorándums.
	\item Varios de los oficios no son revisados y validados por la oficialía de partes.
	\end{itemize}
El proceso de control de oficios y memorándums, actualmente llevado de forma manual, genera problemas al momento de que llega un nuevo oficio al centro. Cuando se le tiene que dar seguimiento a un asunto llevado mediante un oficio no se informa al departamento responsable de atender el asunto y por lo tanto no se sabe el estatus de ésta, además de que la correspondencia con información confidencial en ocasiones es vista por personal no autorizado. A la vez no se cumple con la norma ISO 14000 para la gestión del medio ambiente ya que se desperdicia papel generando copias para informar a los responsables.

		\subsubsection{Problemas con la información}
\begin{itemize}
	
	\item Los empleados del CMPL no siempre se pueden poner en contacto con el personal indicado.
	\item Es complicado consultar las evidencias fotográficas en orden cronológico.
	\item El personal no siempre consulta la información actualizada de los cursos.
	\item No todo el personal se entera de las actividades que se llevarán a cabo en el CMPL.
	\item Se vuelve complicado obtener el material de apoyo.	
	
\end{itemize}

Como podemos ver, estos problemas tienen un impacto
	\subsection{Identificación de causas}
	%Se presentan las causas identificadas y se justifica por qué son causas.
	Los problemas que actualmente existen en el centro son generados debido a ciertas causas en específico. Las causas detectadas en los problemas anteriormente mencionados son:

\begin{itemize}
	
	\item La persona encargada de Oficialía de Partes tiene a su cargo otras actividades.
	\item Los oficios entrantes sólo se registran en la bitácora.
	\item Todos los empleados tienen acceso a la bitácora de registro de los oficios.
	\item El procedimiento de control de oficios y memorándums es manual.
	\item No siempre se registran los oficios y los memorándums en la bitacora.
	\item Se generan oficios con identificadores de oficio iguales.
	\item No se tiene organización del archivo histórico de los oficios entrantes archivados.
	\item No se han actualizado las tecnologías usadas en el centro.
	\item No se cumple con la normatividad de comunicación interna.
	
\end{itemize}
	\subsection{Estimación de consecuencias}
	%Se presentan y justifican las posibles consecuencias a mediano y largo plazo.
	Cuando existen problemas de cualquier tipo en alguna institución, éstos llevan a concecuencias que afectan su funcionamiento o la imágen de la institución. Las concecuancias a corto y mediano plazo que llevan los problemas que actualmente tiene el centro son:

\begin{itemize}
	\item Se ve afectada la comunicación interna y externa del centro.
	\item Se ve afectada la imágen que proyecta el centro a otras instancias.
	\item Anomalías en las auditorias del centro.
	\item Mal uso de la información confidencial.
\end{itemize}
\section{Síntesis de la problemática}
%Se presentan las posibles alternativas de soluciones existentes con sus ventajas y desventajas.

%Se concluye el capítulo presentando las estrategias a implementar y su justificación.



