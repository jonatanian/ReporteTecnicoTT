\section{Problemática}

El proceso de control de correspondencia, actualmente llevado de forma manual, genera problemas al momento de que llega un nuevo oficio al centro. Cuando se le tiene que dar seguimiento a un asunto llevado mediante un oficio no se informa al departamento responsable de atender el asunto y por lo tanto no se sabe el estatus de está, además de que la correspondencia con información confidencial en ocasiones es vista por personal no autorizado. A la vez no se cumple con la norma ISO 14000 para la gestión del medio ambiente ya que se desperdicia papel generando copias para informar a los responsables.

Parte de la problemática detectada en el centro fue detectada con base a las reuniones con el cliente. De las entrevistas de pudieron detectar los problemas que a continuacoón se mencionan:

\begin{itemize}
	
	\item No siempre se le informa al destinatario la llegada de sus oficios o memorádums.
	\item Los oficios con información confidencial es propensa a estar expuesta (violación de la privacidad).
	\item No se sabe el estatus del seguimiento de los oficios.
	\item Es complicado consultar la información de los oficios que están archivados.
	\item El registro de los oficios no se realiza de forma correcta.
	\item Rezago tecnológico.
	\item Se atienden oficios fuera de la fecha límite de respuesta.
	\item No se sabe el desempeño de trabajo del centro respecto a la atención de asuntos con dependecias externas.
	
\end{itemize}

\subsection{Análisis causal}

Los problemas que actualmente existen en el centro son generados debido a ciertas causas en específico. Las causas en específico detectadas en los problemas anteriormente mencionados son:

\begin{itemize}
	
	\item La persona encargada de Oficialía de Partes tiene a su cargo otras actividades.
	\item Los oficios entrantes sólo se registran en la bitácora.
	\item Todos los empleados tienen acceso a la bitácora de registro de los oficios.
	\item El procedimiento de control de oficios y memorándums es manual.
	\item No siempre se registran los oficios y los memorándums en la bitacora.
	\item Se generan oficios con identificadores de oficio iguales.
	\item No se tiene organización del archivo histórico de los oficios entrantes archivados.
	\item No se han actualizado las tecnologías usadas en el centro.
	\item No se cumple con la normatividad de comunicación interna.
	
\end{itemize}

\subsection{Consecuencias a corto y largo plazo}

Cuando existen problemas de cualquier tipo en alguna institución, éstos llevan a concecuencias que afectan su funcionamiento o la imágen de la institución. Las concecuancias a corto y mediano plazo que llevan los problemas que actualmente tiene el centro son:

	\item Se ve afectada la comunicación interna y externa del centro.
	\item Se ve afectada la imágen que proyecta el centro a otras instancias.
	\item Anomalías en las auditorias del centro.
	\item Mal uso de la información confidencial.

