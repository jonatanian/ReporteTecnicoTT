En este capítulo abordamos la problemática que presenta el CMPL en la actualidad. 

%Se describe el contenido del capítulo
En el presente capítulo identificamos, describimos y analizamos la problemática del trabajo terminal. Para realizarlo seguimos un procedimiento de descomposición de la problemática, identificación de causas y la integración de posibles soluciones. Al final del capítulo presentamos la estrategia que se utilizará para el desarrollo del trabajo terminal. 
\section{Planteamiento del problema}
%En un párrafo se presenta y describe la problemática en general con el contexto.
Como mencionamos al final del capítulo anterior, en la actualidad el CMPL realiza todas sus actividades de forma manual. Llevar los procesos de forma manual tiene un impacto negativo en la calidad con la que el CMPL trabaja, ya que resulta complicado llevar los procesos conforme a los manuales. La ausencia de una herramienta de apoyo que reduzca el impacto ambiental en el CMPL, ha sido la causa principal de que el CMPL haya tenido problemas en la obtención de la certificación de la norma ISO 14001. 

La problemática general es que debido a que el CMPL no cuenta con una herramienta de apoyo para la el manejo de oficios y memorándums, la imagen del CMPL se ha visto afectada hacia otras instituciones ya que no se pueden llevar los procedimientos de acuerdo al manual oficial del CMPL, generando problemas en las actividades de la institución que involucra a instituciones o empresas ajenas. Así mismo se tiene problemas con el manejo de información, ésto ha provocado inconsistencias en las tareas que corresponden al CMPL afectando directamente las actividades propias del CMPL y las que tiene con otras instituciones. Parte de los objetivos ambientales del CMPL es reducir el consumo de papel en cierto porcentaje lo cual no se está cumpliendo. Dicho incumplimiento se debe a que se manejan oficios y memorándums de forma manual, gastando copias e impresiones en el procedimiento para le manejo de estos documentos.

\section{Análisis del Problema}
%Se explica el procedimiento para el análisis de la problemática.
La metodología que se utilizó para los problemas que fueron detectados con base a las reuniones con el cliente es:
	\begin{itemize}
		\item Reuniones con el cliente: cada determinado tiempo se organizan reuniones con el cliente, las cueles nos permiten obtener información relevante para el desarrollo de este proyecto.
		\item Obtención de requerimientos: En cada reunión se modifican o actualizan los requerimientos definidos para la aplicación en desarrollo.
	\end{itemize}

Una vez detectados los problemas, describimos cada uno con el fin de identificar sus causas y el impacto que éstos traen al CMPL. Finalmente se van a proponer estrategias a implementar con las posibles soluciones a los problemas. 
	\subsection{Descomposición del problema}
	%Se presentan y describen los problemas uno por uno y se identifica la relación entre ellos.
	A continuación se mencionan los problemas actualmente detectados en el CMPL. Los problemas fueron divididos en dos categorías:
	
		\subsubsection{Problemas con los oficios y memorándums}
\begin{itemize}
	\item No siempre se le informa al destinatario la llegada de sus oficios o memorádums.
Cuando llega un oficio al CMPL, éste es registrado por oficialía de partes y se encarga de entregarlo a la persona indicada. En ocasiones dicho documento no es entregado al personal ya que no siempre la persona a la que le corresponde el oficio se encuentra en su lugar. 

	\item Los oficios con información confidencial es propensa a estar expuesta (violación de la privacidad).
Cuando llega un un oficio al CMPL y es de carácter confidencial o privado, el asunto sólo debe ser leído por oficialía de partes para su registro y también por la persona a la que está dirigido el documento. Actualmente el registro de los oficios y memorándums se lleva en una bitácora que está disponible para todos los empleados del CMPL haciendo propensa la información que se está manejando.

	\item No se sabe el estatus del seguimiento de los oficios.
En ocasiones los oficios y memorándums llevan una fecha límite para ser atendidos, cuando esta fecha es rebasada o está a punto de cumplirse, no se sabe exactamente si se entregó el oficio a la persona correspondiente y si el oficio ya se ha atendido.

	\item Es complicado consultar la información de los oficios que están archivados.
Cuando llega un oficio al CMPL, por el manual de procedimientos debe ser archivado después de que se registra. En ocasiones se necesita consultar el archivo original o información en específico de los oficios archivados pero no se sabe en dónde exactamente está archivado el documento.	
	
	\item El registro de los oficios no se realiza de forma correcta.
Los oficios y memorándums deben de llevar un registro consecutivo, por lo cual el personal debe consultar la bitácora y tomar el identificador que sigue. En ocasiones el personal no registra su nuevo oficio o memorándum ocasionando que se generen registros repetidos.	
	
	\item Rezago tecnológico.
Los servidores que anteriormente alojaban el recurso "Intranet" cuentan con un importante rezago tecnológico ya que no cuentan con los recursos suficientes para ejecutar una aplicación web con tecnologías actuales como HTML5 y SQL Server al no contar con memoria RAM suficiente. De igual forma el almacenamiento es muy limitado al tener un disco duro de 127 GB.
	
	\item Se atienden oficios fuera de la fecha límite de respuesta.
En ocasiones no se registran los oficios en la bitácora con la fecha límite de respuesta ocasionando que no se sepa cuándo hay que dar respuesta al asunto que se debe atender.	
	
	\item No se sabe el desempeño de trabajo del centro respecto a la atención de asuntos con dependecias externas.
Mensualmente se realizan reportes estadísticos para saber el desempeño de las áreas del CMPL en cuanto a las actividades que le concierne. Actualmente no se pueden realizar dichos reportes para saber cual fue el cumplimiento de los asuntos atendidos mediante oficios y memorandums con precisión debido a que actualmente existen inconsistencias en el registro y el seguimiento de estos documentos.
	
	\item Es complicado saber la ubicación física de algunos oficios o memorándums. %Explicación
Debido a que todos los oficios y memorándums deben ser archivados, cuando se requiere el documento original para verificar información requerida no se puede ya que la docuemnto está extraviado o fue transpapelado accidentalmente.
	
	\item Varios de los oficios no son revisados y validados por la oficialía de partes.
Los oficios salientes realizados para dar respuesta a un asunto, deben de llevar un formato en específico que debe ser validado por oficialía de partes. En ocasiones dicho formato no es revisado y es firmado así por el director provocando un incumplimiento en el formato estándar definido por el CMPL.	
	\end{itemize}
	
El proceso de control de oficios y memorándums, actualmente llevado de forma manual, genera problemas al momento de que llega un nuevo oficio al centro. Cuando se le tiene que dar seguimiento a un asunto llevado mediante un oficio no se informa al departamento responsable de atender el asunto y por lo tanto no se sabe el estatus de ésta, además de que la correspondencia con información confidencial en ocasiones es vista por personal no autorizado. A la vez no se cumple con la norma ISO 14000 para la gestión del medio ambiente ya que se desperdicia papel generando copias para informar a los responsables.

		\subsubsection{Problemas con el manejo de la información}
\begin{itemize}
	
	\item Los empleados del CMPL no siempre se pueden poner en contacto con el personal indicado.	
El CMPL debe de tener en el SIG el directorio actualizado del los trabajadores que laboran el la institución. Cuando requieren ponerse en contacto con alguna persona en específico, en ocasiones no se lleva con éxito esta comunicación ya que el directorio se encuentra desactualizado y no se puede modificar de una forma rápida y sencilla.
	
	\item Es complicado consultar las evidencias fotográficas en orden cronológico.
Parte del SIG es tener evidencias fotográficas de los eventos realizados por el CMPL. Dichas evidencias deben estar disponibles para ser consultadas pero actualmente no es así ya que no se cuenta con una herramienta que permita tener dichas evidencias disponibles y organizadas.
	
	\item El personal no siempre consulta la información actualizada de los cursos.
En ocasiones el CMPL imparte cursos relacionados al rol de la institución para sus trabajadores pero la información que actualmente se encuentra en el recurso de la Intranet está desactualizada, haciendo que no sirva está información generando pérdidas económicas al no ser impartidos con éxito los cursos.

	\item No todo el personal se entera de las actividades que se llevarán a cabo en el CMPL.
Cuando el CMPL requiere avisar a sus trabajadores de avisos o cambios que se llevarán a cabo, no siempre se logra el aviso con éxito ya que muchos de los trabajadores se encuentran trabajando fuera de su lugar como en los laboratorios o en otras instituciones o empresas.	
	
	\item Se vuelve complicado obtener el material de apoyo.	
Cuando el CMPL imparte la maestría o cursos, requieren material de apoyo que no está organizado debido a que no se encuentra almacenado en un lugar en específico ocasionando problemas con el personal que se encuentra tanto impartiendo los cursos como el personal que toma el curso.	
	
\end{itemize}

Como podemos ver, estos problemas tienen un impacto
	\subsection{Identificación de causas}
	%Se presentan las causas identificadas y se justifica por qué son causas.
	Los problemas que actualmente existen en el centro son generados debido a ciertas causas en específico. Las causas detectadas en los problemas anteriormente mencionados son:

\begin{itemize}
	
	\item La persona encargada de Oficialía de Partes tiene a su cargo otras actividades.
La persona encargada de oficialía de partes es la secretaria del director, por lo que no siempre atiende los oficios cuando llegan al CMPL, es decir, no los registra de forma correcta y en ocasiones no tiene tiempo o no está en condiciones de entregar los oficios a la persona que debe atenderlo.

	\item Los oficios entrantes sólo se registran en la bitácora.
Cuando llega un oficio, éste es registrado a mano en una bitácora, que también lleva los registros de los oficios salientes y memorándums internos. No siempre se lleva un correcto registro de la correspondencia de la bitácora e incluso se llega a perder todo este registro generando problemas en las actividades del CMPL ya que no se tenía un respaldo de la información de registro de todos los oficios y memorándums.
	\item Todos los empleados tienen acceso a la bitácora de registro de los oficios.
	
	\item El procedimiento de control de oficios y memorándums es manual.
Cuando llega un oficio o se crea un memorándum en el CMPL, éste asunto de tiene que llevar de forma manual (con el oficio original que pasa por los responsables) y por tanto conlleva a que se saquen copias del documento para que sea atendido por el personal correspondiente. De igual forma lo hace propenso a que el oficio o memorándum sea extraviado, transpapelado o que incluso se arruine por accidente. Cuando el original de un oficio o memorándum ya no puede ser usado por alguna de las causas mencionadas, el documento tiene que ser impreso de nuevo o se tienen que sacar más copias del mismo haciendo que se vaya en contra de la norma ISO 14000.

	\item No siempre se registran los oficios y los memorándums en la bitacora.
Debido a que la encargada de Oficialía de Partes tiene otras actividades a su cargo ya que es la secretaria del director del CMPL, no siempre atiende los oficios y memorándums de una forma correcta empezando desde su registro. 

	\item Se generan oficios con identificadores de oficio iguales.	
Cada que se crea un oficio o memorándum en el CMPL, el personal tiene que tomar un número consecutivo de registro de la bitácora, pero no siempre se registra en la bitácora, haciendo propenso a que otra persona tome un número consecutivo pensando que tomo el correcto. Cuando Oficialía de Partes tiene que validar los documentos para que sean firmados y sellados por el director, tiene que cambiar el identificador consecutivo de los oficios ya que no pueden existir oficios o memorándums con el mismo identificador; ocasionando a la par	que se tenga que volver a gastar papel para la impresión de los oficios o memorándums por corrección del número consegutivo.

	\item No se tiene organización del archivo histórico de los oficios entrantes archivados.
Cuando un oficio es atendido, posteriormente debe ser archivado. Actualmente se encuentra una gran cantidad de oficios archivados, lo cual genera un problema en el espacio reservado para estos documentos cuando se tiene que buscar un oficio en caso de que se requiera consultar información en el documento.
	
	\item No se han actualizado las tecnologías usadas en el CMPL.
Las tecnologías actualmente disponibles en el CMPL están limitadas para un buen almacenamiento de información. Los servidores actualmente disponibles  para el almacenamiento cuentan con discos duros de aproximadamente 127 GB de almacenamiento, el cual ya se encuentra con más del 80\% de su capacidad haciendo que estos servidores no estén disponibles para su uso como servidores de almacenamiento o de aplicaciones.

	\item No se cumple con la normatividad de comunicación interna.
Debe de existir una herramienta de comunicación interna propia del CMPL de acuerdo al manual de procedimientos. Dicha herramienta su función es evitar el uso de papel para generan avisos y promover la participación social de los trabajadores. Cuando se llevó a cabo la última auditoria en el CMPL existieron faltas ya que no existe dicha herramienta.
	
\end{itemize}
	\subsection{Estimación de consecuencias}
	%Se presentan y justifican las posibles consecuencias a mediano y largo plazo.
	Cuando existen problemas de cualquier tipo en alguna institución, éstos llevan a concecuencias que afectan su funcionamiento o la imágen de la institución. Las concecuancias a corto y mediano plazo que llevan los problemas que actualmente tiene el centro son:

\begin{itemize}
	\item Se ve afectada la comunicación interna y externa del centro.	
	\item Se ve afectada la imágen que proyecta el centro a otras instancias.
	\item Anomalías en las auditorias del centro.
	\item Mal uso de la información confidencial.
	\item Pérdida de la certificación ISO 9001.
	\item Pérdida de la certificación ISO 14000.
\end{itemize}
\section{Síntesis de la problemática}
%Se presentan las posibles alternativas de soluciones existentes con sus ventajas y desventajas.

%Se concluye el capítulo presentando las estrategias a implementar y su justificación.



