\section{Problemática}
El CMPL tiene su propio sistema de control de la calidad llamado Sistema Integrado de Gestión de la Calidad y del Ambiente (SIG). El objetivo del centro es cumplir con la política integral de calidad y del ambiente del CMPL, la cual define que:

\begin{quote}
El Centro Mexicano para la Producción más Limpia ofrece asistencia técnica y programas de formación de recursos humanos de excelencia, comprometido con la satisfacción de los clientes con productos diseñados bajo criterios ambientales y de calidad  que cumplan la legislación ambiental y otros requerimientos  aplicables, teniendo como ejes centrales la prevención de la contaminación y la mejora continua.
\end{quote}

Para el cumplimento de esta política el centro, a través de toda su organización,  estableció los siguientes principios y compromisos:

\begin{itemize}
	\item	Llevar a cabo un proceso de mejora continua en todos los ámbitos a través del establecimiento y revisión de objetivos y metas.
	\item 	Tener en cuenta los requisitos establecidos por nuestros clientes
	\item 	Asumir el compromiso de cumplir los requisitos aplicables, tanto legales y reglamentarios como otros que la organización suscriba.
	\item 	Implicar, motivar y comprometer a todo el personal para que se involucre en la organización, así como su formación, motivación y comunicación
	\item 	Desarrollar actividades formativas para que todos los integrantes del CMP+L conozcan, participen y apliquen el Programa de Protección Civil del IPN.
	\item 	Establecer como uno de nuestros objetivos principales la prevención de la contaminación. 	\item Utilizar de modo racional, oportuno, pertinente y adecuado los recursos materiales, fomentar el ahorro energético y la reducción de la producción de residuos. 
\end{itemize}

El centro tiene sus objetivos de calidad y de ambiente definidos. Dichos objetivos son definidos con base a los principios y compromisos de la política de calidad y del ambiente.

Objetivos de calidad:

\begin{itemize}
	\item Supervisar el cumplimiento del plan operativo anual.
	\item Impulsar la maestría en Ingeniería de producción más limpia.
	\item Supervisar el cumplimiento con los lineamientos administrativos definidos por el Instituto.
	\item Impulsar las actividades de investigación, desarrollo e innovación (I+D+i) de proyectos nacionales o internacionales de producción más limpia y los relacionados con el desarrollo industrial sustentable.
 \item Promover la vinculación con el Gobierno, la iniciativa privada a nivel nacional y la vinculación internacional.
	\item Fortalecer el Programa Institucional para la Sustentabilidad del IPN.
\end{itemize}

Objetivos ambientales:

\begin{itemize}
	\item Reducir el consumo de agua un 2.5\%.
	\item Reducir un 3\% para el 2014, respecto al consumo de energía eléctrica del 2012.
	\item Reducir el consumo de papel un 5\%.
	\item Mantener el Programa para la Gestión Integral de Residuos.
	\item Realizar el mantenimiento del transformador cada dos años.
\end{itemize}

Es indispensable que se tenga una correcta administración de los documentos en general del centro, ya que se tiene especificado cómo es que se debe de llevar a cabo este procedimiento en el manual de procedimientos del centro, contemplando que es una insitución de producción más limpia y que a la par cumple con la normatividad ambiental.




%Se describe el contenido del capítulo
En el presete capítulo identificamos, describimos y analisamos la problemática del trabajo terminal. Para realizarlo seguimos un procedimiento de descomposición de la problemática, identificación de causas y la integración de posibles soluciones. Al final del capítulo presentamos la estrategia que se utilizará para el desarrollo del trabajo terminal. 
\section{Planteamiento del problema}
%En un párrafo se presenta y describe la problemática en general con el contexto.
\section{Análisis del Problema}
%Se explica el procedimiento para el análisis de la problemática.
	\subsection{Descomposición del problema}
	%Se presentan y describen los problemas uno por uno y se identifica la relación entre ellos.
	\subsection{Identificación de causas}
	%Se presentan las causas identificadas y se justifica por qué son causas.
	\subsection{Estimación de consecuencias}
	%Se presentan y justifican las posibles consecuencias a mediano y largo plazo.
\section{Síntesis de la problemática}
%Se presentan las posibles alternativas de soluciones existentes con sus ventajas y desventajas.

%Se concluye el capítulo presentando las estrategias a implementar y su justificación.




El proceso de control de oficios y memorándums, actualmente llevado de forma manual, genera problemas al momento de que llega un nuevo oficio al centro. Cuando se le tiene que dar seguimiento a un asunto llevado mediante un oficio no se informa al departamento responsable de atender el asunto y por lo tanto no se sabe el estatus de ésta, además de que la correspondencia con información confidencial en ocasiones es vista por personal no autorizado. A la vez no se cumple con la norma ISO 14000 para la gestión del medio ambiente ya que se desperdicia papel generando copias para informar a los responsables.

Parte de la problemática en el centro fue detectada con base a las reuniones con el cliente. De las entrevistas de pudieron detectar los problemas que a continuacoón se mencionan:

\begin{itemize}
	
	\item No siempre se le informa al destinatario la llegada de sus oficios o memorádums.
	\item Los oficios con información confidencial es propensa a estar expuesta (violación de la privacidad).
	\item No se sabe el estatus del seguimiento de los oficios.
	\item Es complicado consultar la información de los oficios que están archivados.
	\item El registro de los oficios no se realiza de forma correcta.
	\item Rezago tecnológico.
	\item Se atienden oficios fuera de la fecha límite de respuesta.
	\item No se sabe el desempeño de trabajo del centro respecto a la atención de asuntos con dependecias externas.
	
\end{itemize}

\subsection{Análisis causal}

Los problemas que actualmente existen en el centro son generados debido a ciertas causas en específico. Las causas en específico detectadas en los problemas anteriormente mencionados son:

\begin{itemize}
	
	\item La persona encargada de Oficialía de Partes tiene a su cargo otras actividades.
	\item Los oficios entrantes sólo se registran en la bitácora.
	\item Todos los empleados tienen acceso a la bitácora de registro de los oficios.
	\item El procedimiento de control de oficios y memorándums es manual.
	\item No siempre se registran los oficios y los memorándums en la bitacora.
	\item Se generan oficios con identificadores de oficio iguales.
	\item No se tiene organización del archivo histórico de los oficios entrantes archivados.
	\item No se han actualizado las tecnologías usadas en el centro.
	\item No se cumple con la normatividad de comunicación interna.
	
\end{itemize}

\subsection{Consecuencias a corto y largo plazo}

Cuando existen problemas de cualquier tipo en alguna institución, éstos llevan a concecuencias que afectan su funcionamiento o la imágen de la institución. Las concecuancias a corto y mediano plazo que llevan los problemas que actualmente tiene el centro son:

\begin{itemize}
	\item Se ve afectada la comunicación interna y externa del centro.
	\item Se ve afectada la imágen que proyecta el centro a otras instancias.
	\item Anomalías en las auditorias del centro.
	\item Mal uso de la información confidencial.
\end{itemize}
