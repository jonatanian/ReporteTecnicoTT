\section{Trabajo Futuro}

%poner lo de los reportes mensuales (indicadores)
%poner la generacion del pdf desde el wizard sin necesidad de subirlo
%poner lo de el reconocimiento de patrones 

Este Trabajo Terminal se desarrolló con base en la normatividad del Centro implementando una nueva propuesta de software  para el control de correspondencia mediante el uso de nuevas herramientas que no se habían utilizado antes. La creación de la aplicación permite un escalamiento para los futuros que pueda llegar a tener o las futuras aplicaciones que se desarrollen. Uno de los cambios que se podrían hacer es: \\

\begin{itemize}
	\item La generación de un pdf. En está parte ya no sería necesario adjuntar el documento al final de redactarlo, se podría generar el documento en pdf al final de que llenaron todos los campos correspondientes al formulario. 
	\item Reconocimiento de imagenes. Esta parte sería más complicada de realizar porque sólo se registraría que hay un nuevo oficio entrante pero al momento en que se escaneé el oficio y se guarde como imagen se haría el reconocimiento de la imagen para sacar todos los datos del oficio.
\end{itemize}

En este trabajo se abordó el tema de la correspondencia que, implica el uso del papel para hacer duplicado del original más de una vez.Por lo tanto se espera que esta aplicación se utilice en otras instancias del IPN para formar un sistema en conjunto que reduzca el uso del papel ayudando a el medio ambiente pero que además sirva de apoyo a el personal del IPN a tener un mejor control del estatus de sus oficios y memos mejorando los indicadores que llevan cada mes.\\
