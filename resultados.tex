\section{Resultados Obtenidos}

Actualmente se tiene el desarrollo y queda pendiente la implementación en el servidor del CMPL, cabe mencionar que en las computadoras que se utilizan como servidores de pruebas la aplicación funciona adecuadamente y se puede realizar lo siguiente: 
 \begin{itemize}
 	\item Registrar un oficio cuando llega. 
 	Esto a su vez puede realizar lo siguiente: 
 	\begin{itemize}
 		\item Registrar la dependencia de donde se envía en caso de que no exista.
 		\item El nombre del emisor y su cargo.
 		\item El área que emite el oficio. 
 		\item Número del oficio.
 		\item Fecha de emisión y de recepción.
 		\item Registrar los anexos. 
 		\item Anexar la copia escaneada del oficio.
 	\end{itemize}
 	\item Que el oficio que se registra se turne en automatico al director.
 	\item Que el director seleccione el tipo de oficio y con base en eso turnar a la o las personas responsables de atenderlo.
 	\item Una vez turnado, si es jefe de departamento o subdirector pueden volver a turnar el oficio a el personal.
 	\item Si el director se equivocó al momento de que turnó el oficio, el personal puede notificar que no es para él.
 	\item Atender el oficio.
 	\item Oficialía puede dar seguimiento, es decir, ver el estado del oficio.
 	\item El personal puede atender el oficio. 
 	\item Se puede registrar un nuevo oficio ya sea de respuesta o saliente. 
 	\item Consultar los oficios registrados por parte de oficialia de partes, y por parte del personal consultar los oficios que le han sido turnados, cuales ya atendió y cuales tiene pendientes.
 	\item Filtrar los oficios por su id, dependencia y área. 
 	\item Guardar el acuse de recibido, subiendo una copía escaneada. 
 	\item Que el director autorice el oficio a imprimir.
 	\item Registrar observaciones de un oficio.
 	\item Subir un oficio con las observaciones.
 	\item Generar el id del oficio cuando se responde y cuando se registra uno nuevo.
 \end{itemize} 
 
Con base en lo anterior se cumple que se tenga un respaldo de los oficios, se dé un mejor seguimiento, se lleve u orden al responder, se cumpla con la normatividad para la certificación, se emitan oficios bien hechos. \\

Por el momento se no se pueden generar los oficios dentro de la aplicación pero se espera que este TT se continue con eso y la generación de pdf. En un futuro se espera que se puedan obtener los datos del oficio que se registra mediante el escaneo del documento y con el uso de reconocimiento de imagenes y patrones se llenen los campos. 

%funciona no funciona que sí se puede hacer  que no 
%que quedo a futuro 
%que fue lo más dificil
%que errores tuvimos
%que aciertos