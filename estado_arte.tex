¿Qué es correspondencia?

La correspondencia "es el género de composición que comprende las distintas formas de comunicación escrita, en los trámites mercantiles, industriales y oficiales" [6].

La correspondencia "en su aceptación tradicional se refiere a las cartas y toda clase de documentos y toda clase de documentos que se despachen por correo y, por extensión, a las comunicaciones radiotelegráficas" [7].

La correspondencia se divide en:

* Correspondencia comercial.

* Correspondencia industrial.

* Correspondencia institucional.

Cada tipo de correspondencia posee normas para su funcionamiento, las cuales es necesario conocer.

Dicho lo anterior y en este caso solo se tratará la correspondencia institucional.

"La correspondencia institucional se refiere a asuntos relacionados con el Estado o entidades no oficiales, cuyo carácter no es comercial. Esta puede ser correspondencia dirigida del gobierno a particulares, de particulares a oficinas estatales o entre dependencia y funcionarios públicos" [7].

\section{Estado del arte}

Hoy en día 
Una aplicación para el seguimiento de egresados es una herramienta muy común hoy en día, existen diferentes instituciones públicas o privadas en las cuales es importante para ellos conocer que ha pasado con sus egresados a lo largo de los años, por lo que recurren a este tipo de aplicaciones para tener un contacto más cercano con ellos. Cada una tiene características especiales dependiendo de las carreras que ofrecen, por ejemplo como se había comentado antes el SISAE que brinda seguimiento a todo el IPN.

En la actualidad existen algunos sistemas que cuentan con ciertas características y funcionalidades similares a las que se proponen para el manejo de la correspondencia. Dichos sistemas se describen a continuación.