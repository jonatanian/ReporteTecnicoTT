\section{Contexto}

\subsection{El centro}

El Centro Mexicano para la Producción más Límpia (en adelante, CMPL) es una instancia del Instituto Politécnico Nacional que se dedica a la investigación y elaboración de procesos de producción más limpia. El CMPL, como dice en su página web, “es integrante de la red mundial de centros de producción más límpia y de la red latinoamericana de producción más limpia, promovidas por la Organización de las Naciones Unidas para el Desarrollo Industrial (ONUDI) y el Programa de las Naciones Unidas para el Medio Ambiente (PNUMA). Cuenta con 13 años de experiencia realizando trabajo técnico para la industria nacional atendiendo sectores como alimentos, petroquímicos, cementeros, galvanoplastia y embotelladoras, por mencionar algunos. Los servicios que ofrece el CMPL son: diagnóstico en producción más limpia y eficiencia energética, diplomados a distancia, presenciales y Maestría en Producción Más Limpia, realización de proyectos de mecanismo de desarrollo limpio, planes de manejo de residuos y análisis de químicos” [1].\\

El CMPL consta de varios departamentos, dentro de los cuales  se encuentra el Departamento de Sistemas y Banco de Datos encargado de sistematizar, desarrollar y automatizar los procesos y/o actividades del CMPL. Así mismo también se encuentra la persona encargada de llevar el ---
