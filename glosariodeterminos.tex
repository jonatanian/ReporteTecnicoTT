\section{Glosario de Terminos}

\subsubsection{Definiciones}

\begin{itemize}
	\item Administrador: persona que tiene el control sobre el sistema.
	\item Correspondencia: Es la correspondencia formal (interna y externa) que trabaja el CMPL. La correspondencia formal interna es conocida como memorándums y la correspondencia formal externa como oficios.
	\item Iteración: es una técnica de desarrollar y entregar componentes de funcionalidades del negocio.
	\item Servidor: aplicación en ejecución capaz de atender las peticiones de un cliente y devolverle una respuesta en concordancia.
	\item Cliente: aquellos individuos u organizaciones que están activamente involucrados en el proyecto, o cuyos intereses pueden verse afectados, positiva o negativamente, como resultado de la ejecución y término del proyecto. \cite{cliente}
	\item Consulta: método para acceder a los datos en una base de datos.
	\item Base de Datos: conjunto de datos pertenecientes a un mismo contexto y almacenados sistemáticamente para uso posterior.
	\item Pruebas de validación: proceso de revisión para comprobar que el proyecto cumple con sus especificaciones.
	\item Conexión: comunicación entre dos entes que tienen características similares de comunicación.
	\item Oficialía de partes: persona encarga de llevar el control de oficios recibidos y emitidos a través del sistema.
	\item Usuario: persona que utiliza el sistema.
\end{itemize}

\subsubsection{Acrónimos}

\begin{itemize}
	\item CMPL: Centro Mexicano para la Producción más Limpia.
	\item TIC: Tecnologías de la Información y Comunicación.
	\item SISA: Sistema de Administración.
\end{itemize}
