\section{Conclusiones}

El diseño e implementación de esta aplicación para el manejo de la correspondencia no fue una tarea fácil debido a que en este caso fue necesario tener un entendimiento de la normatividad que se utiliza primeramente en el IPN y después, el manual de procedimientos del CMPL y fue este el que nos llevo a revisar los estándares sobre los que trabaja que son ISO 9001 e ISO 14000. \\

Durante el transcurso del desarrollo del trabajo terminal, uno de los retos más complicados fueron las reuniones con el cliente porque en las reuniones que se tuvieron no sabía lo que en realidad quería por lo que la obtención de requerimientos, ideas y necesidades fue un poco complicada y con base en eso hacer las sugerencias necesarias diciendo que no necesita y que sí necesita. Otro factor importante fue la negociación con el Centro porque muchas veces pasó que lo que ellos creen que es algo simple de hacer para nosotros como programadores y diseñadores es muy complicado de hacer.\\

Una de las partes más difíciles de desarrollar fue la parte de los estados por los que pasa un oficio porque implicó el uso de banderas en diferentes perfiles de usuario. Sin embargo, nos llevamos una experiencia profesional grata porque aprendimos más de lo que esperabamos y cumplimos con el objetivo de mitigar las causas y los impactos a los problemas que se presentaban en el Centro.