\documentclass[oneside,10pt]{book}

\usepackage{cdtBook}
\usepackage{usecases}

\title{Reporte Técnico}
\subtitle{Aplicación web de Administración y Control de Correspondencia para el CMPL del IPN (SISA-CMPL)}
\author{Trabajo Terminal No. 2014-B060}
%\organization{Escuela Superior de Cómputo, IPN}


%%%%%%%%%%%%%%%%%%%%%%%%%%%%%%%%%%%%%%%%%%%%%%%%%%%%%%%%%%%%%%%%
\begin{document}

\maketitle
\thispagestyle{empty}

\frontmatter
\tableofcontents

\mainmatter

%=========================================================
\chapter{Antecedentes}
	En esta sección hablaremos acerca de lo que es el Centro Mexicano para la Producción más Limpia, cuál es su función como parte del IPN y las normas y procedimientos que lo rigen, así como la estructura interna que maneja para la realización y coordinación de sus diversas actividades.
\cfinput{contexto}
	
\cfinput{problematica}

\cfinput{propuesta}

%=========================================================
\chapter{Análisis}

%\cfinput{}

%=========================================================
\chapter{Diseño}

%\cfinput{}


%=========================================================
\chapter{Modelo de Negocios}

\cfinput{glosariodeterminos}

\cfinput{reglasdenegocio}

\cfinput{estados}

%=========================================================
\chapter{Modelo de Información}

%\cfinput{basedatos}

\cfinput{diccionariodatos}
%=========================================================
%\chapter{Modelo de Casos de Uso}
	
%\cfinput{cu17/cu}

%%=========================================================
%\chapter{Modelo de la Interacción}

%{\color{UCInterfaceColor} 
%	Esta sección se queda deliberadamente en blanco debido a que el diseño de las interfaces dependerá de la plataforma a utilizar por cada equipo.\\	
%}

%\cfinput{Pantallas/IU23}


%=========================================================
%\chapter{Modelo del Dominio del problema}

	
\end{document}
